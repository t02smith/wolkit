
\section{Wake-on-LAN}

Wake-on-LAN is a network standard that allows remote devices to be turned on by sending a \textit{magic packet} to it. It was first introduced in 1998~\cite{noauthor_ibm_2012} by the Advanced Manageability Alliance AMA; this alliance consists of several large technology companies that was created with the purpose of creating standards that helped streamline computer management.
Some examples of standards introduced by the AMA are: Desktop Management Interface~\cite{noauthor_dmi_nodate}, Alert Standard Format~\cite{noauthor_asf_nodate}, and Common Information Model~\cite{noauthor_cim_nodate}.

\subsection{Magic Packet}

A magic packet is a frame with a 102 byte payload that consists of: 6 bytes that are all \textit{0xff}, and 16 copies of the target device's MAC address. The target device will be listening specifically for this packet whilst in a low-power mode; when it is broadcast over the network the target device's network interface card will send a command to the power supply or motherboard to wake the system up.

\subsection{Limitations}

The WoL standard is limited by:

\vspace{-2mm}
\begin{itemize}[noitemsep]
  \item not having any form of delivery confirmation so there is no way to tell if the device was actually woken without interfacing with it directly,
  \item requiring knowledge of the device's MAC address to be able to wake it, and
  \item being limited to IP-based devices.
\end{itemize}