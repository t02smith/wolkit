\section{Introduction}

It is a common scenario for home devices to be left running all day despite only being in use for a fraction of the time they are awake~\cite{urban_energy_nodate}. The Wake-on-LAN standard offers the ability for many devices to be left in low-power modes and woken using a single central node; by watching for several different events in the local network and environment this project will be able to wake devices exclusively when they are required. By having this central node as a low-power device, this will result in a potentially large reduction in power usage.
\x
Some example use-cases of this application are:

\begin{itemize}[noitemsep]
    \item Scheduling devices to be woken at recurring periods when a device is needed to be used. This allows the user to have a predictable timetable for when the device should be active.
    \item Waking a device when the presence of another device is detected locally. The presence of a local device can infer that the user is nearby and will want to use a different device.
    \item Similarly, waking a device based upon environmental factors can infer the presence of a user but does not require the user to have a mobile device that would need to be detected.
\end{itemize}
