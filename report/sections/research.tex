
\section{Research}

\subsection{Wake-on-LAN}

Wake-on-LAN is a network standard that allows remote devices to be turned on by sending a \textit{magic packet} to it. It was first introduced in 1998\footnote{\url{https://web.archive.org/web/20121012155338/http://www-03.ibm.com/press/us/en/pressrelease/2705.wss} (Accessed 09/05/2023)} by the Advanced Manageability Alliance (AMA); this alliance consisted of several large tech companies and was created with the purpose of introducing standards that would helped streamline computer management.
Some examples of standards introduced by the AMA are: Desktop Management Interface\footnote{\url{https://www.dmtf.org/standards/dmi} (Accessed 09/05/2023)}, Alert Standard Format\footnote{\url{https://www.dmtf.org/standards/asf} (Accessed 09/05/2023)}, and Common Information Model\footnote{\url{https://www.dmtf.org/standards/cim} (Accessed 09/05/2023)}.

\vspace{2mm}
\subsubsection{Magic Packet}

A magic packet is a frame with a 102 byte payload that consists of: 

\begin{itemize}[noitemsep]
    \item 6 bytes that are all \textit{0xff}, and
    \item 16 copies of the target device's MAC address.
\end{itemize}

The target device's network interface card (NIC) will be listening for this packet whilst in a low-power mode; when it is broadcast over the network the target device's NIC will send a command to the power supply or motherboard to wake the system up.

\vspace{2mm}
\subsubsection{Limitations}

The Wake-on-LAN standard doesn't include any form of delivery confirmation so there is now way to know if the magic packet sent is received or acted upon. 

\subsection{Waking Non IP-Based Device}

Wake-on-LAN uses an IP packet and as such cannot be sent to devices that don't use IP; these are commonly IoT devices and will use protocols like Zigbee to communicate. Han et al looked at using Zigbee to control power outlets in their paper \textit{Remote-controllable and energy-saving room architecture based on ZigBee communication}~\cite{han_remote-controllable_2009}; devices can be toggled using an IR remote or will automatically be turned off when the power output from the socket falls below a certain threshold.

\subsection{Device Discovery}

\vspace{2mm}
\subsubsection{Over Wireless LAN}

If we are aware of the device's IP address then it is trivial to discover if it is active by sending it a simple request, such as an ARP request, and wait for a response.
\x
Zhou et al describe a system called \textit{ZiFi}~\cite{zhou_zifi_2010}, which uses Zigbee radios to identify the existence of local WiFi networks; this was with the aim of reducing the power requirement needed for a device to discover local WiFi networks. Whilst the use of Zigbee is promising for device discovery, most devices will not come fitted with Zigbee radios. 

\vspace{2mm}
\subsubsection{Bluetooth}

Bluetooth is a short-range, low-bandwidth, low-power wireless communication protocol that is commonly found in battery powered devices. Bluetooth devices organise themselves into \textit{piconets}, which consist of one \textit{master} and up to seven \textit{slave} devices. Each device in the same piconet will have the same frequency hopping sequence that will determined by the master. The formation of a piconet has two steps:

\begin{enumerate}[noitemsep]
    \item \textbf{Inquiry} A master device will discover neighbouring slave devices, and
    \item \textbf{Page} Connections are established between devices.
\end{enumerate}

During the inquiry phase, the master will broadcast an inquiry packet and scan for replies. If a slave wants to be discovered then it will periodically scan for inquiry packets and send a response if they find one. In their paper \textit{A formal analysis of bluetooth discovery}~\cite{duflot_formal_2006}, Duflot et al give a review of the performance of device discovery in Bluetooth v1.2 and v1.1.
A similar analysis of Bluetooth 4.0 is given by Cho et al in their paper \textit{Performance analysis of device discovery of Bluetooth Low Energy (BLE) networks}~\cite{cho_performance_2016}.
\x
Cross et al showed it was possible to discover devices that are not scanning for and replying to inquiry packets in their paper \textit{Detecting non-discoverable Bluetooth devices}~\cite{goetz_detecting_2007}. This was achieved using an `enhanced brute force search attack`; this relied on already knowing the device's MAC address but would still take a considerable amount of time, where most Bluetooth devices would take around 18 hours. 
